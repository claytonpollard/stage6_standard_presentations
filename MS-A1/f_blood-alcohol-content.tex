\documentclass[aspectratio=169,10pt]{beamer}

\usefonttheme[onlymath
]{serif}

\usetheme[progressbar=head,numbering=none]{metropolis}
\beamertemplatenavigationsymbolsempty
%%\setbeamertemplate{background}[grid]

\usepackage{tasks}
\usepackage{cancel}
\usepackage{multicol}
\usepackage{mathtools}

\definecolor{col1}{HTML}{3A86FF}
\definecolor{col2}{HTML}{0BBF7D}
\definecolor{col3}{HTML}{FFBE0B}
\definecolor{col4}{HTML}{FF006E}
\definecolor{col5}{HTML}{733907}
\definecolor{col6}{HTML}{B340D7}

\setbeamercolor{progress bar}{fg=col2,bg=col3}
\setbeamercolor{title}{fg=col1!40!black}
\setbeamercolor{frametitle}{bg=col2!40!black, fg=white}

\makeatletter
\setlength{\metropolis@progressinheadfoot@linewidth}{2pt}
\setlength{\metropolis@titleseparator@linewidth}{2pt}
\setlength{\metropolis@progressonsectionpage@linewidth}{2pt}
\makeatother
% Boxes
\usepackage[most]{tcolorbox} % Required for boxes
	\tcbuselibrary{skins,breakable,xparse}
\usepackage{varwidth}

\newcounter{definition}
\resetcounteronoverlays{definition}

\renewtcolorbox[auto counter]{definition}[1][]{standard jigsaw,enhanced,sharp corners,frame hidden,boxrule=0pt,breakable,colback=col1!20!white,fonttitle=\bfseries,coltitle=col1!50!black,colframe=col1!50!white,title=Definition~\thedefinition\quad#1\newline,attach title to upper,borderline west={2pt}{0pt}{col1!80!black},left=3mm,phantom={\global\refstepcounter{definition}}}

\newcounter{example}
\resetcounteronoverlays{example}

\renewtcolorbox[auto counter]{example}[1][]{standard jigsaw,enhanced,sharp corners,frame hidden,boxrule=0pt,breakable,colback=col2!20!white,fonttitle=\bfseries,coltitle=col2!50!black,colframe=col2!50!white,title=Example~\theexample\quad#1\newline,attach title to upper,borderline west={2pt}{0pt}{col2!80!black},left=3mm,phantom={\global\refstepcounter{example}}}

\renewtcolorbox{solution}[1][height=4cm]{standard jigsaw,enhanced,sharp corners,frame hidden,boxrule=0pt,breakable,#1,colback=col2!80!white,fonttitle=\bfseries,coltitle=col2!50!black,colframe=col2!50!white,opacityback=.1,title=Solution\newline,attach title to upper,borderline west={2pt}{0pt}{col2!80!black},left=3mm,top=3mm,,borderline north={1pt}{0pt}{col2!80!black},before={\vspace{-7pt}}}

\renewtcolorbox{note}{standard jigsaw,enhanced,sharp corners,frame hidden,boxrule=0pt,breakable,colback=col3!20!white,fonttitle=\normalfont\bfseries,coltitle=col3!50!black,colframe=col3!50!white,title=Note~,attach title to upper,borderline west={2pt}{0pt}{col3!80!black},left=3mm,fontupper=\itshape}

\newtcolorbox{important}{standard jigsaw,enhanced,sharp corners,frame hidden,boxrule=0pt,breakable,colback=col4!20!white,fonttitle=\bfseries,coltitle=col4!50!black,colframe=col4!50!white,title=Important Note\newline,attach title to upper,borderline west={2pt}{0pt}{col4!80!black},left=3mm}

\newtcolorbox{further}{standard jigsaw,enhanced,sharp corners,frame hidden,boxrule=0pt,breakable,colback=col5!20!white,fonttitle=\bfseries,coltitle=col5!50!black,colframe=col5!50!white,title=Further Exercises\newline,attach title to upper,borderline west={2pt}{0pt}{col5!80!black},left=3mm}

\newtcolorbox{outcome}{standard jigsaw,enhanced,sharp corners,frame hidden,boxrule=0pt,breakable,colback=col6!20!white,fonttitle=\bfseries,coltitle=col6!50!black,colframe=col6!50!white,title=Learning Outcome\newline,attach title to upper,borderline west={2pt}{0pt}{col6!80!black},left=3mm}

\newtcolorbox{law}[1][]{standard jigsaw,enhanced,sharp corners,frame hidden,boxrule=0pt,breakable,colback=col3!20!white,fonttitle=\bfseries,coltitle=col1!50!black,colframe=col1!50!white,title=#1~Law\newline,attach title to upper,borderline west={2pt}{0pt}{col1!80!black},borderline north={2pt}{0pt}{col1!80!black},left=3mm,top=3mm}

\renewtcolorbox{theorem}[1][]{standard jigsaw,enhanced,sharp corners,frame hidden,boxrule=0pt,breakable,colback=col3!20!white,fonttitle=\bfseries,coltitle=col2!50!black,colframe=col2!50!white,title=#1~Theorem\newline,attach title to upper,borderline west={2pt}{0pt}{col2!80!black},borderline north={2pt}{0pt}{col2!80!black},left=3mm,top=3mm}

\newtcolorbox{result}{standard jigsaw,enhanced,sharp corners,frame hidden,boxrule=0pt,breakable,colback=col3!20!white,fonttitle=\bfseries,coltitle=col4!50!black,colframe=col4!50!white,title=Result\newline,attach title to upper,borderline west={2pt}{0pt}{col4!80!black},borderline north={2pt}{0pt}{col4!80!black},left=3mm,top=3mm}

\renewtcolorbox{proof}[1][height=4cm]{standard jigsaw,enhanced,sharp corners,frame hidden,boxrule=0pt,breakable,#1,colback=col3!80!white,fonttitle=\bfseries,coltitle=col4!50!black,colframe=col3!50!white,opacityback=.1,title=Proof\newline,attach title to upper,borderline west={2pt}{0pt}{col4!80!black},left=3mm,top=3mm,borderline north={1pt}{0pt}{col4!80!black},before={\vspace{-7pt}}}

\usepackage{environ}

\title{Blood Alcohol Content}
\subtitle{Standard}
\author{MS-A1 Formulae and Equations}
\usepackage[style=iso]{datetime2}
\date{updated: \today
}

\begin{document}

\begin{frame}{Revision}
\textbf{General 2 2018 Q28e}\\
    Sophie is driving at 70 km/h. She notices a branch on the road ahead and decides to apply the brakes. Her reaction time is 1.5 seconds. Her braking distance ($D$ metres) is given by $D=0.01v^2$, where $v$ is speed in km/h.
    
    What is Sophie's stopping distance, to the nearest metre? \hfill\textbf{3}\vspace{2em}\pause
  \begin{solution}[]
\pause$78$ m
    \end{solution}
\end{frame}

\frame{\titlepage}

\begin{frame}
  \begin{outcome}
    \textbf{Topic:}
    \begin{itemize}
      \item[] Blood Alcohol Content
    \end{itemize}

    \textbf{Syllabus:}
    \begin{itemize}
      \item calculate and interpret blood alcohol content (BAC) based on drink consumption and body weight
      \begin{itemize}
          \item use formulae, both in word form and algebraic form, to calculate an estimate for blood alcohol content (BAC), including $BAC_{Male}=\dfrac{10N−7.5H}{6.8M}$ and $BAC_{Female}=\dfrac{10N−7.5H}{5.5M}$ where $N$ is the number of standard drinks consumed, $H$ is the number of hours of drinking, and $M$ is the person’s weight in kilograms.
          \item determine the number of hours required for a person to stop consuming alcohol in order to reach zero BAC, eg using the formula $\text{time}=\dfrac{BAC}{0.015}$
          \item describe the limitations of methods estimating BAC
      \end{itemize}
    \end{itemize}

    \textbf{Activities/Tasks:}
    \begin{itemize}
      \item Cambridge Ex 3F Q1-11
    \end{itemize}
  \end{outcome}
\end{frame}

\begin{frame}{Blood Alcohol Content}
  \pause
  \begin{definition}
    Blood alcohol content (BAC) is a measure of the amount of alcohol in your blood.
  \end{definition}\pause
  BAC tells you the number of grams of alcohol per 100 mL of blood.\pause
  
  For example a BAC of 0.10 means there is 0.10 g of alcohol for every 100 mL of blood, or 1 g for every 1 L of blood.
\end{frame}

\begin{frame}{Estimating BAC}
    BAC can be estimated using the following formulae:
    \begin{formula}
    $$BAC_{\text{Male}}=\frac{10N-7.5H}{6.8M}\quad\text{or}\quad BAC_{\text{Female}}=\frac{10N-7.5H}{5.5M}$$
  \begin{columns}
    \begin{column}{.34\textwidth}
      Where:
      \begin{itemize}
        \item[$BAC$] - Blood alcohol content\pause
        \item[$N$] - Number of standard drinks consumed\pause
        \item[$H$] - Hours drinking\pause
        \item[$M$] - Mass in kilograms
      \end{itemize}
    \end{column}
    \begin{column}{.54\textwidth}
      \begin{important}
        These formulae are \textbf{not} on the reference sheet.
        
        You are \textbf{not} expected to remember these formulae.
      \end{important}
    \end{column}
  \end{columns}
    \end{formula}
\end{frame}

\begin{frame}{Limitations}
    These formulae are only good for calculating \textit{estimates} of a person BAC.\pause

    The formulae for estimating BAC only takes into account the number of standard drinks consumed, sex, the amount of time, and mass.\pause
    There are other factors that influence BAC including:
    \begin{itemize}
        \item Fitness
        \item Health
        \item Liver function
        \item Food in the stomach
        \item Medications
    \end{itemize}
\end{frame}

\begin{frame}
  \begin{example}
    Osman is 87 kg and has consumed 3 standard drinks in the past hour. Estimate Osman's BAC to 3 decimak places using the following formula where $N$ represents the number of standard drinks, $H$ is the number of hours drinking and $M$ is his mass in kilograms.
    $$BAC_{\text{Male}}=\frac{10N-7.5H}{6.8M}$$
  \end{example}\pause
  \begin{solution}[]
\[
\begin{aligned}
   BAC_{\text{Male}}&=\frac{10N-7.5H}{6.8M}\\\pause
   &=\frac{10(3)-7.5(1)}{6.8(87)}\\\pause
   &=0.038032...\\\pause
   &\approx0.038
\end{aligned}
\]
Osman's BAC is estimated to be 0.038.
  \end{solution}
\end{frame}

\begin{frame}{Reaching zero BAC}
    The number of hours required for a person to reach zero BAC after they stop consuming alcohol is given by the following formula:\pause
    
    \begin{formula}
    $$\text{Time (h)}=\frac{BAC}{0.015}$$\pause

      \begin{important}
        This formula is \textbf{not} on the reference sheet.
        
        You are \textbf{not} expected to remember this formula.
      \end{important}

    \end{formula}
\end{frame}

\begin{frame}
  \begin{example}
    Fiona has a BAC of 0.027. Use the following formula to estimate the time she will need for her BAC to reach zero. Answer to the nearest minute.
    $$\text{Time}=\frac{BAC}{0.015}$$
  \end{example}\pause
  \begin{solution}[]
    \[
      \begin{aligned}
        \text{Time}&=\frac{BAC}{0.015}\\\pause
        &=\frac{(0.027)}{0.015}\\\pause
        &=1.8\\\pause
        &\quad\text{\casio{x}}\\\pause
        &=1^\circ48'0''\pause
      \end{aligned}
      \]
      Fiona needs to wait 1 hour and 48 minutes.
  \end{solution}
\end{frame}

\begin{frame}{Today's work}
  \begin{itemize} 
    \item Cambridge Ex 3F Q1-11
  \end{itemize}
\end{frame}

\end{document}