\documentclass[aspectratio=169,10pt]{beamer}

\usefonttheme[onlymath
]{serif}

\usetheme[progressbar=head,numbering=none]{metropolis}
\beamertemplatenavigationsymbolsempty
%%\setbeamertemplate{background}[grid]

\usepackage{tasks}
\usepackage{cancel}
\usepackage{multicol}
\usepackage{mathtools}

\definecolor{col1}{HTML}{3A86FF}
\definecolor{col2}{HTML}{0BBF7D}
\definecolor{col3}{HTML}{FFBE0B}
\definecolor{col4}{HTML}{FF006E}
\definecolor{col5}{HTML}{733907}
\definecolor{col6}{HTML}{B340D7}

\setbeamercolor{progress bar}{fg=col2,bg=col3}
\setbeamercolor{title}{fg=col1!40!black}
\setbeamercolor{frametitle}{bg=col2!40!black, fg=white}

\makeatletter
\setlength{\metropolis@progressinheadfoot@linewidth}{2pt}
\setlength{\metropolis@titleseparator@linewidth}{2pt}
\setlength{\metropolis@progressonsectionpage@linewidth}{2pt}
\makeatother
% Boxes
\usepackage[most]{tcolorbox} % Required for boxes
	\tcbuselibrary{skins,breakable,xparse}
\usepackage{varwidth}

\newcounter{definition}
\resetcounteronoverlays{definition}

\renewtcolorbox[auto counter]{definition}[1][]{standard jigsaw,enhanced,sharp corners,frame hidden,boxrule=0pt,breakable,colback=col1!20!white,fonttitle=\bfseries,coltitle=col1!50!black,colframe=col1!50!white,title=Definition~\thedefinition\quad#1\newline,attach title to upper,borderline west={2pt}{0pt}{col1!80!black},left=3mm,phantom={\global\refstepcounter{definition}}}

\newcounter{example}
\resetcounteronoverlays{example}

\renewtcolorbox[auto counter]{example}[1][]{standard jigsaw,enhanced,sharp corners,frame hidden,boxrule=0pt,breakable,colback=col2!20!white,fonttitle=\bfseries,coltitle=col2!50!black,colframe=col2!50!white,title=Example~\theexample\quad#1\newline,attach title to upper,borderline west={2pt}{0pt}{col2!80!black},left=3mm,phantom={\global\refstepcounter{example}}}

\renewtcolorbox{solution}[1][height=4cm]{standard jigsaw,enhanced,sharp corners,frame hidden,boxrule=0pt,breakable,#1,colback=col2!80!white,fonttitle=\bfseries,coltitle=col2!50!black,colframe=col2!50!white,opacityback=.1,title=Solution\newline,attach title to upper,borderline west={2pt}{0pt}{col2!80!black},left=3mm,top=3mm,,borderline north={1pt}{0pt}{col2!80!black},before={\vspace{-7pt}}}

\renewtcolorbox{note}{standard jigsaw,enhanced,sharp corners,frame hidden,boxrule=0pt,breakable,colback=col3!20!white,fonttitle=\normalfont\bfseries,coltitle=col3!50!black,colframe=col3!50!white,title=Note~,attach title to upper,borderline west={2pt}{0pt}{col3!80!black},left=3mm,fontupper=\itshape}

\newtcolorbox{important}{standard jigsaw,enhanced,sharp corners,frame hidden,boxrule=0pt,breakable,colback=col4!20!white,fonttitle=\bfseries,coltitle=col4!50!black,colframe=col4!50!white,title=Important Note\newline,attach title to upper,borderline west={2pt}{0pt}{col4!80!black},left=3mm}

\newtcolorbox{further}{standard jigsaw,enhanced,sharp corners,frame hidden,boxrule=0pt,breakable,colback=col5!20!white,fonttitle=\bfseries,coltitle=col5!50!black,colframe=col5!50!white,title=Further Exercises\newline,attach title to upper,borderline west={2pt}{0pt}{col5!80!black},left=3mm}

\newtcolorbox{outcome}{standard jigsaw,enhanced,sharp corners,frame hidden,boxrule=0pt,breakable,colback=col6!20!white,fonttitle=\bfseries,coltitle=col6!50!black,colframe=col6!50!white,title=Learning Outcome\newline,attach title to upper,borderline west={2pt}{0pt}{col6!80!black},left=3mm}

\newtcolorbox{law}[1][]{standard jigsaw,enhanced,sharp corners,frame hidden,boxrule=0pt,breakable,colback=col3!20!white,fonttitle=\bfseries,coltitle=col1!50!black,colframe=col1!50!white,title=#1~Law\newline,attach title to upper,borderline west={2pt}{0pt}{col1!80!black},borderline north={2pt}{0pt}{col1!80!black},left=3mm,top=3mm}

\renewtcolorbox{theorem}[1][]{standard jigsaw,enhanced,sharp corners,frame hidden,boxrule=0pt,breakable,colback=col3!20!white,fonttitle=\bfseries,coltitle=col2!50!black,colframe=col2!50!white,title=#1~Theorem\newline,attach title to upper,borderline west={2pt}{0pt}{col2!80!black},borderline north={2pt}{0pt}{col2!80!black},left=3mm,top=3mm}

\newtcolorbox{result}{standard jigsaw,enhanced,sharp corners,frame hidden,boxrule=0pt,breakable,colback=col3!20!white,fonttitle=\bfseries,coltitle=col4!50!black,colframe=col4!50!white,title=Result\newline,attach title to upper,borderline west={2pt}{0pt}{col4!80!black},borderline north={2pt}{0pt}{col4!80!black},left=3mm,top=3mm}

\renewtcolorbox{proof}[1][height=4cm]{standard jigsaw,enhanced,sharp corners,frame hidden,boxrule=0pt,breakable,#1,colback=col3!80!white,fonttitle=\bfseries,coltitle=col4!50!black,colframe=col3!50!white,opacityback=.1,title=Proof\newline,attach title to upper,borderline west={2pt}{0pt}{col4!80!black},left=3mm,top=3mm,borderline north={1pt}{0pt}{col4!80!black},before={\vspace{-7pt}}}

\usepackage{environ}

\title{Speed, distance and time}
\subtitle{Standard}
\author{MS-A1 Formulae and Equations}
\usepackage[style=iso]{datetime2}
\date{updated: \today
}

\begin{document}

\begin{frame}{Revision}
\begin{tasks}(1)
    \task \textbf{General 2 2005 Q24c}\\
    Make $L$ the subject of the equation $\ \ T=2\pi L^2$.\hfill\textbf{2}
    \task \textbf{General 2 2017 Q28d}\\
    Make $y$ the subject of the equation $\ \ x=\sqrt{yp-1}$.\hfill\textbf{2}
    \task \textbf{Standard 1 2019 Q34}\\
    Given the formula $\ \ C=\dfrac{A(y+1)}{24},\ \ $ calculate the value of $y$ when $C=120$ and $A=500$.\hfill\textbf{3}
\end{tasks}\vspace{7pt}\pause
  \begin{solution}[]
    \begin{tasks}(3)
    \task \pause$L=\pm\sqrt{\dfrac{T}{2\pi}}$
    \task \pause$y=\dfrac{x^2+1}{p}$
    \task \pause$y=4.76$
    \end{tasks}
    \end{solution}
\end{frame}

\frame{\titlepage}

\begin{frame}
  \begin{outcome}
    \textbf{Topic:}
    \begin{itemize}
      \item[] Speed, distance and time
    \end{itemize}

    \textbf{Syllabus:}
    \begin{itemize}
      \item solve problems involving formulae, including calculating distance, speed and time (with change of units of measurement as required) or calculating stopping distances of vehicles using a suitable formula
    \end{itemize}

    \textbf{Activities/Tasks:}
    \begin{itemize}
      \item Cambridge Ex 3B Q1-20
    \end{itemize}
  \end{outcome}
\end{frame}

\begin{frame}{Speed}
  \begin{definition}
  \textbf{Speed} is a comparison between \textit{distance travelled} and \textit{time taken}.
  \end{definition}\pause
  
  In most real world situations, an objects speed is not constant. Therefore we often use the \textbf{average speed}.\pause
  
  \begin{formula}
    $$s=\frac{d}{t}$$\pause
  \begin{columns}
    \begin{column}{.34\textwidth}
      Where:
      \begin{itemize}
        \item[$s$] - average speed\pause
        \item[$d$] - distance travelled\pause
        \item[$t$] - time taken
      \end{itemize}
    \end{column}
    \begin{column}{.54\textwidth}
      \begin{important}
        This formula is \textbf{not} on the reference sheet.
      \end{important}
    \end{column}
  \end{columns}
  \end{formula}
\end{frame}

\begin{frame}
  \begin{example}
    Find the average speed of a car which travels 110 km in 2 hours.
  \end{example}\pause
  \begin{solution}[]
    \[
    \begin{aligned}
      s&=\frac{d}{t}\\\pause
      &=\frac{110\text{ km}}{2\text{ h}}\\\pause
      &= 55\text{ km/h}
    \end{aligned}
    \]
  \end{solution}
\end{frame}

\begin{frame}{Rearranging}
  The formula for speed can rewritten to make \textit{distance} or \text{time} the subject.
  $$s=\frac{d}{t}\quad\quad d=st\quad\quad t=\frac{d}{s}$$
\end{frame}

\begin{frame}
  \begin{example}
    Jonah rides his motorcycle on a highway at an average speed of 90 km/h.
    \begin{tasks}(1)
      \task How far can Jonah travel in $1\frac{1}{2}$ hours?
      \task How long will he take to travel 210 km? Answer in hours and minutes.
    \end{tasks}
  \end{example}\pause
  \begin{solution}[]
  \vspace{-1em}
  \begin{columns}[t]
      \begin{column}{.44\textwidth}
        a)\vspace{-1em}
        \[
        \begin{aligned}
          d&=st\\\pause
          d&=\underset{\text{\textcolor{col4}{{\casio qa1\$1R2}}}}{(90)\left(\textcolor{col4}{1\frac{1}{2}}\right)}\\\pause
          d&=135\text{ km}
        \end{aligned}
        \]
      \end{column}
      \begin{column}{.44\textwidth}
        b)\vspace{-1em}
        \[
        \begin{aligned}
          t&=\frac{d}{s}\\\pause
          t&=\frac{210}{90}\\\pause
          t&=2.33...\text{ h}\\\pause
          &\quad\text{\casio x}\\\pause
          t&=2^\circ 20' 0''
        \end{aligned}
        \]
        2 hours and 20 minutes.
      \end{column}
  \end{columns}
  \end{solution}
\end{frame}

\begin{frame}{Stopping distance}
    \begin{definition}[Stopping distance]
    The distance a vehicle will travel in order to come to a complete stop.
    \end{definition}\pause
    $$\text{stopping distance}=\text{reaction distance}+\text{braking distance}$$\\\pause\vspace{1em}
    \begin{important}
    You \textbf{are not} expected to memorise any formulae.\pause
    
    You \textbf{are} expected to be able to substitute into a given formula.
    \end{important}
\end{frame}

\begin{frame}
    \begin{example}
      Claire is driving on a motorway at a speed of 110 kilometres per hour and has to brake suddenly. She has a reaction time of 2 seconds and a braking distance of 59.2 metres.

      What is Claire's stopping distance?
    \end{example}\pause
    \begin{solution}[]\pause
    \begin{columns}[t]
    \begin{column}{.3\textwidth}
    \[
    \begin{aligned}
      110\text{ km/h}&=110\,000\text{ m/hr}\\\pause
      &=\frac{110\,000}{60\times60}\text{ m/s}\\\pause
      &=30.555...\text{ m/s}\pause
    \end{aligned}
    \]
    \end{column}
    \begin{column}{.6\textwidth}
    Let $d$ be Claire's stopping distance.\pause
    \[
    \begin{aligned}
      d&=\underbrace{\text{reaction time distance}}_{d=st}+\text{braking distance}\\\pause
      &=st+\text{braking distance}\\\pause
      &=(30.555...)(2)+(59.2)\\\pause
      &=120.311...\\\pause
      d&=120.3 \text{ m (1 d.p.)}
      \end{aligned}\pause
    \]
    \end{column}
    \end{columns}
    Claire's stopping distance is 120.3 metres.
    \end{solution}
\end{frame}

\begin{frame}
    \begin{example}
      Max was driving 60 km/h and has a reaction time of 0.8 s. Calculate the stopping distance correct to the nearest metre given the formula $\quad d=\dfrac{5vt}{18}+\dfrac{v^2}{170}\quad$ where $d$ is the stopping distance (m), $v$ is the speed (m/s) and $t$ is reaction time (s).
    \end{example}\pause
    \begin{solution}[]
    \[
    \begin{aligned}
      d&=\frac{5vt}{18}+\frac{v^2}{170}\\\pause
      d&=\frac{5(60)(0.8)}{18}+\frac{60^2}{170}\\\pause
      d&=34.509...\\\pause
      d&\approx35\text{ m}\pause
    \end{aligned}
    \]
    Max's stopping distance is 35 metres.
    \end{solution}
\end{frame}

\begin{frame}{Today's work}
  \begin{itemize} 
    \item Cambridge Ex 3B Q1-20
  \end{itemize}
\end{frame}

\end{document}