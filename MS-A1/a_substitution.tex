\documentclass[aspectratio=1610,10pt]{beamer}

\usefonttheme[onlymath
]{serif}

\usetheme[progressbar=head,numbering=none]{metropolis}
\beamertemplatenavigationsymbolsempty
%%\setbeamertemplate{background}[grid]

\usepackage{tasks}
\usepackage{cancel}
\usepackage{multicol}
\usepackage{mathtools}

\definecolor{col1}{HTML}{3A86FF}
\definecolor{col2}{HTML}{0BBF7D}
\definecolor{col3}{HTML}{FFBE0B}
\definecolor{col4}{HTML}{FF006E}
\definecolor{col5}{HTML}{733907}
\definecolor{col6}{HTML}{B340D7}

\setbeamercolor{progress bar}{fg=col2,bg=col3}
\setbeamercolor{title}{fg=col1!40!black}
\setbeamercolor{frametitle}{bg=col2!40!black, fg=white}

\makeatletter
\setlength{\metropolis@progressinheadfoot@linewidth}{2pt}
\setlength{\metropolis@titleseparator@linewidth}{2pt}
\setlength{\metropolis@progressonsectionpage@linewidth}{2pt}
\makeatother
% Boxes
\usepackage[most]{tcolorbox} % Required for boxes
	\tcbuselibrary{skins,breakable,xparse}
\usepackage{varwidth}

\newcounter{definition}
\resetcounteronoverlays{definition}

\renewtcolorbox[auto counter]{definition}[1][]{standard jigsaw,enhanced,sharp corners,frame hidden,boxrule=0pt,breakable,colback=col1!20!white,fonttitle=\bfseries,coltitle=col1!50!black,colframe=col1!50!white,title=Definition~\thedefinition\quad#1\newline,attach title to upper,borderline west={2pt}{0pt}{col1!80!black},left=3mm,phantom={\global\refstepcounter{definition}}}

\newcounter{example}
\resetcounteronoverlays{example}

\renewtcolorbox[auto counter]{example}[1][]{standard jigsaw,enhanced,sharp corners,frame hidden,boxrule=0pt,breakable,colback=col2!20!white,fonttitle=\bfseries,coltitle=col2!50!black,colframe=col2!50!white,title=Example~\theexample\quad#1\newline,attach title to upper,borderline west={2pt}{0pt}{col2!80!black},left=3mm,phantom={\global\refstepcounter{example}}}

\renewtcolorbox{solution}[1][height=4cm]{standard jigsaw,enhanced,sharp corners,frame hidden,boxrule=0pt,breakable,#1,colback=col2!80!white,fonttitle=\bfseries,coltitle=col2!50!black,colframe=col2!50!white,opacityback=.1,title=Solution\newline,attach title to upper,borderline west={2pt}{0pt}{col2!80!black},left=3mm,top=3mm,,borderline north={1pt}{0pt}{col2!80!black},before={\vspace{-7pt}}}

\renewtcolorbox{note}{standard jigsaw,enhanced,sharp corners,frame hidden,boxrule=0pt,breakable,colback=col3!20!white,fonttitle=\normalfont\bfseries,coltitle=col3!50!black,colframe=col3!50!white,title=Note~,attach title to upper,borderline west={2pt}{0pt}{col3!80!black},left=3mm,fontupper=\itshape}

\newtcolorbox{important}{standard jigsaw,enhanced,sharp corners,frame hidden,boxrule=0pt,breakable,colback=col4!20!white,fonttitle=\bfseries,coltitle=col4!50!black,colframe=col4!50!white,title=Important Note\newline,attach title to upper,borderline west={2pt}{0pt}{col4!80!black},left=3mm}

\newtcolorbox{further}{standard jigsaw,enhanced,sharp corners,frame hidden,boxrule=0pt,breakable,colback=col5!20!white,fonttitle=\bfseries,coltitle=col5!50!black,colframe=col5!50!white,title=Further Exercises\newline,attach title to upper,borderline west={2pt}{0pt}{col5!80!black},left=3mm}

\newtcolorbox{outcome}{standard jigsaw,enhanced,sharp corners,frame hidden,boxrule=0pt,breakable,colback=col6!20!white,fonttitle=\bfseries,coltitle=col6!50!black,colframe=col6!50!white,title=Learning Outcome\newline,attach title to upper,borderline west={2pt}{0pt}{col6!80!black},left=3mm}

\newtcolorbox{law}[1][]{standard jigsaw,enhanced,sharp corners,frame hidden,boxrule=0pt,breakable,colback=col3!20!white,fonttitle=\bfseries,coltitle=col1!50!black,colframe=col1!50!white,title=#1~Law\newline,attach title to upper,borderline west={2pt}{0pt}{col1!80!black},borderline north={2pt}{0pt}{col1!80!black},left=3mm,top=3mm}

\renewtcolorbox{theorem}[1][]{standard jigsaw,enhanced,sharp corners,frame hidden,boxrule=0pt,breakable,colback=col3!20!white,fonttitle=\bfseries,coltitle=col2!50!black,colframe=col2!50!white,title=#1~Theorem\newline,attach title to upper,borderline west={2pt}{0pt}{col2!80!black},borderline north={2pt}{0pt}{col2!80!black},left=3mm,top=3mm}

\newtcolorbox{result}{standard jigsaw,enhanced,sharp corners,frame hidden,boxrule=0pt,breakable,colback=col3!20!white,fonttitle=\bfseries,coltitle=col4!50!black,colframe=col4!50!white,title=Result\newline,attach title to upper,borderline west={2pt}{0pt}{col4!80!black},borderline north={2pt}{0pt}{col4!80!black},left=3mm,top=3mm}

\renewtcolorbox{proof}[1][height=4cm]{standard jigsaw,enhanced,sharp corners,frame hidden,boxrule=0pt,breakable,#1,colback=col3!80!white,fonttitle=\bfseries,coltitle=col4!50!black,colframe=col3!50!white,opacityback=.1,title=Proof\newline,attach title to upper,borderline west={2pt}{0pt}{col4!80!black},left=3mm,top=3mm,borderline north={1pt}{0pt}{col4!80!black},before={\vspace{-7pt}}}

\usepackage{environ}

\title{Substitution}
\subtitle{Standard}
\author{MS-A1 Formulae and Equations}
\usepackage[style=iso]{datetime2}
\date{updated: \today
}

\begin{document}

\frame{\titlepage}

\begin{frame}
  \begin{outcome}
    \textbf{Topic:}
    \begin{itemize}
      \item[] Substitution
    \end{itemize}

    \textbf{Syllabus:}
    \begin{itemize}
      \item review substitution of numerical values into linear and non-linear algebraic expressions and equations
    \end{itemize}

    \textbf{Activities/Tasks:}
    \begin{itemize}
      \item Cambridge Ex 3A Q1-17
    \end{itemize}
  \end{outcome}
\end{frame}

\begin{frame}{Why algebra?}
  \pause
  \textbf{Algebra} is used Mathematicians to communicate mathematical ideas in a convenient way. \textbf{Variables} (letters and symbols) are used to represent unknown quantities whose value can change depending on the situation.
\end{frame}

\begin{frame}{Expressions}
  We can think of expressions as number crunching machines. Numbers are  put in and then a related number is produced.\pause

  For example the expression $\ a^2+1\ $ starts with an input $a$, squares it and adds 1.\pause

  If the number -5 is fed into the machine, the machine \textbf{substitutes} the number 5 in place of $a$ and then \textbf{evaluates} the result:\pause

  \[
    \begin{aligned}
         &a^2+1\\\pause
      =\ &(\textcolor{col4}{-5})^2+1\\\pause
      =\ &25+1\\\pause
      =\ &26
    \end{aligned}
    \]
\end{frame}

\begin{frame}{Substituting}
  \begin{important}
    To avoid making mistakes it is good practice to \textbf{always} substitue into brackets.
  \end{important}
\end{frame}

\begin{frame}
  \begin{example}
    Evaluate the following, given $x=4$
    \begin{tasks}(2)
      \task $2x-3$
      \task $\sqrt{6x+12}$
    \end{tasks}
  \end{example}\pause
  \begin{solution}[]
    \vspace{-1em}
    \begin{columns}[t]
      \begin{column}{.4\textwidth}
    a) $ $\vspace{-1em}\[
      \begin{aligned}
         &2x-3\\\pause
      =\ &2(\textcolor{col4}{4})-3\\\pause
      =\ &8-3\\\pause
      =\ &5\pause
      \end{aligned}
      \]
    \end{column}
    \begin{column}{.4\textwidth}
    b) $ $\vspace{-1em}\[
      \begin{aligned}
         &\sqrt{6x+12}\\\pause
      =\ &\sqrt{6(\textcolor{col4}{4})+12}\\\pause
      =\ &\sqrt{24+12}\\\pause
      =\ &\sqrt{36}\\\pause
      =\ &6\\
      \end{aligned}
      \]
    \end{column}
  \end{columns}
  \end{solution}
\end{frame}

\begin{frame}
  \begin{example}
    Evaluate $\quad a^2-5b+c\quad$ given $a=3$, $b=-3$, $c=7$
  \end{example}\pause
  \begin{solution}[]
    \[
      \begin{aligned}
         &a^2-5b+c\\\pause
      =\ &(\textcolor{col4}{3})^2-5(\textcolor{col4}{-3})+(\textcolor{col4}{7})\\\pause
      =\ &9+15+7\\\pause
      =\ &31
      \end{aligned}
      \]
  \end{solution}
\end{frame}

\begin{frame}{Today's work}
  \begin{itemize} 
    \item Cambridge Ex 3A Q1-17
  \end{itemize}
\end{frame}

\end{document}