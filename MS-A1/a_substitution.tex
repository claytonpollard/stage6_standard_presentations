\documentclass[aspectratio=1610,10pt]{beamer}

\input{./include/structure.tex}
\input{./include/environments.tex}

\title{Substitution}
\subtitle{Standard}
\author{MS-A1 Formulae and Equations}
\usepackage[style=iso]{datetime2}
\date{updated: \today
}

\begin{document}

\frame{\titlepage}

\begin{frame}
  \begin{outcome}
    \textbf{Topic:}
    \begin{itemize}
      \item[] Substitution
    \end{itemize}

    \textbf{Syllabus:}
    \begin{itemize}
      \item review substitution of numerical values into linear and non-linear algebraic expressions and equations
    \end{itemize}

    \textbf{Activities/Tasks:}
    \begin{itemize}
      \item Cambridge Ex 3A Q1-17
    \end{itemize}
  \end{outcome}
\end{frame}

\begin{frame}{Why algebra?}
  \pause
  \textbf{Algebra} is used Mathematicians to communicate mathematical ideas in a convenient way. \textbf{Variables} (letters and symbols) are used to represent unknown quantities whose value can change depending on the situation.
\end{frame}

\begin{frame}{Expressions}
  We can think of expressions as number crunching machines. Numbers are  put in and then a related number is produced.\pause

  For example the expression $\ a^2+1\ $ starts with an input $a$, squares it and adds 1.\pause

  If the number -5 is fed into the machine, the machine \textbf{substitutes} the number 5 in place of $a$ and then \textbf{evaluates} the result:\pause

  \[
    \begin{aligned}
         &a^2+1\\\pause
      =\ &(\textcolor{col4}{-5})^2+1\\\pause
      =\ &25+1\\\pause
      =\ &26
    \end{aligned}
    \]
\end{frame}

\begin{frame}{Substituting}
  \begin{important}
    To avoid making mistakes it is good practice to \textbf{always} substitue into brackets.
  \end{important}
\end{frame}

\begin{frame}
  \begin{example}
    Evaluate the following, given $x=4$
    \begin{tasks}(2)
      \task $2x-3$
      \task $\sqrt{6x+12}$
    \end{tasks}
  \end{example}\pause
  \begin{solution}[]
    \vspace{-1em}
    \begin{columns}[t]
      \begin{column}{.4\textwidth}
    a) $ $\vspace{-1em}\[
      \begin{aligned}
         &2x-3\\\pause
      =\ &2(\textcolor{col4}{4})-3\\\pause
      =\ &8-3\\\pause
      =\ &5\pause
      \end{aligned}
      \]
    \end{column}
    \begin{column}{.4\textwidth}
    b) $ $\vspace{-1em}\[
      \begin{aligned}
         &\sqrt{6x+12}\\\pause
      =\ &\sqrt{6(\textcolor{col4}{4})+12}\\\pause
      =\ &\sqrt{24+12}\\\pause
      =\ &\sqrt{36}\\\pause
      =\ &6\\
      \end{aligned}
      \]
    \end{column}
  \end{columns}
  \end{solution}
\end{frame}

\begin{frame}
  \begin{example}
    Evaluate $\quad a^2-5b+c\quad$ given $a=3$, $b=-3$, $c=7$
  \end{example}\pause
  \begin{solution}[]
    \[
      \begin{aligned}
         &a^2-5b+c\\\pause
      =\ &(\textcolor{col4}{3})^2-5(\textcolor{col4}{-3})+(\textcolor{col4}{7})\\\pause
      =\ &9+15+7\\\pause
      =\ &31
      \end{aligned}
      \]
  \end{solution}
\end{frame}

\begin{frame}{Today's work}
  \begin{itemize} 
    \item Cambridge Ex 3A Q1-17
  \end{itemize}
\end{frame}

\end{document}