\documentclass[aspectratio=169,10pt]{beamer}

\input{./include/structure.tex}
\input{./include/environments.tex}

\title{Formulae}
\subtitle{Standard}
\author{MS-A1 Formulae and Equations}
\usepackage[style=iso]{datetime2}
\date{updated: \today
}

\begin{document}

\begin{frame}{Revision}
\begin{tasks}(1)
    \task \textbf{General 2 2014 Q26c}\\
    Solve the equation $\dfrac{5x+1}{3}-4=5-7x$.\hfill\textbf{3}
    \task \textbf{General 2 2018 Q28b}\\
    Solve the equation $\quad\dfrac{2x}{5}+1=\dfrac{3x+1}{2},\quad$ leaving your answer as a fraction.\hfill\textbf{3}
\end{tasks}\vspace{7pt}\pause
  \begin{solution}[]
    \begin{tasks}(2)
    \task \pause$x=1$
    \task \pause$x=\dfrac{5}{11}$
    \end{tasks}
    \end{solution}
\end{frame}

\frame{\titlepage}

\begin{frame}
  \begin{outcome}
    \textbf{Topic:}
    \begin{itemize}
      \item[] Formulae
    \end{itemize}

    \textbf{Syllabus:}
    \begin{itemize}
      \item review evaluating the subject of a formula, given the value of other pronumerals in the formula
    \end{itemize}

    \textbf{Activities/Tasks:}
    \begin{itemize}
      \item Cambridge Ex 3D Q1-27
    \end{itemize}
  \end{outcome}
\end{frame}

\begin{frame}{Formulae}
  \begin{definition}[]
    A \textbf{formula} is an equation which connects two or more variables. The plural of formula is \textbf{formulae} or \textbf{formulas}.
  \end{definition}\pause
  Generally a formula is written with a single variable, the \textbf{subject} on the LHS and the rest on the right.
  $$\underset{\overset{\uparrow}{\text{subject}}}{A}=\frac{1}{2}ab\cos{C}$$
\end{frame}

\begin{frame}{Formula substitution}
    If the formula contains two or more variables and we know the value of all but one of them, we can solve an equation to find the remaining variable.\pause
        {\setlength\leftmargini{7ex}
    \begin{enumerate}
    	\item[\textit{Step 1}:] State the variables in the formula and the known values.\pause
    	\item[\textit{Step 2}:] Write down the formula and Substitute the known values into the formula.\pause
    	\item[\textit{Step 3}:] Solve the one variable equation to find the unknown value.
    \end{enumerate}}
\end{frame}

\begin{frame}
  \begin{example}
    The area of a trapezium is given by $\ \ A=\dfrac{h}{2}(a+b)\ \ $ where $a$ and $b$ are the lengths of the parallel sides and $h$ is the height.
    
    Calculate the height of a 20 cm$^2$ trapezium with parallel lengths of 3 cm and 7 cm.
  \end{example}\pause
  \begin{solution}[]
    \vspace{-3em}
    \begin{columns}[t]
      \begin{column}{.3\textwidth}
      \[
      \begin{aligned}
        A&=\onslide<3->{20\text{ cm}^2}\\
        h&=\onslide<4->{?}\\
        a&=\onslide<5->{3\text{ cm}}\\
        b&=\onslide<6->{7\text{ cm}}
      \end{aligned}\pause\pause\pause\pause\pause
      \]
    \end{column}
    \begin{column}{.5\textwidth}
      \[
      \begin{aligned}
        A&=\frac{h}{2}(a+b)\\\pause
        20&=\frac{h}{2}(3+7)\\\pause
        20&=\frac{h}{2}(10)\\\pause
        20&=5h\\\pause
        20\textcolor{col4}{\,\div\,5}&=5h\textcolor{col4}{\,\div\,5}\\\pause
        4&=h\\\pause
        h&=4\pause\text{ cm}
      \end{aligned}
      \]
    \end{column}
  \end{columns}
  The height of the trapezium is equal to 4 cm.
  \end{solution}
\end{frame}

\begin{frame}{Today's work}
  \begin{itemize} 
    \item Cambridge Ex 3D Q1-27
  \end{itemize}
\end{frame}

\end{document}