\documentclass[aspectratio=169,10pt]{beamer}

\usefonttheme[onlymath
]{serif}

\usetheme[progressbar=head,numbering=none]{metropolis}
\beamertemplatenavigationsymbolsempty
%%\setbeamertemplate{background}[grid]

\usepackage{tasks}
\usepackage{cancel}
\usepackage{multicol}
\usepackage{mathtools}

\definecolor{col1}{HTML}{3A86FF}
\definecolor{col2}{HTML}{0BBF7D}
\definecolor{col3}{HTML}{FFBE0B}
\definecolor{col4}{HTML}{FF006E}
\definecolor{col5}{HTML}{733907}
\definecolor{col6}{HTML}{B340D7}

\setbeamercolor{progress bar}{fg=col2,bg=col3}
\setbeamercolor{title}{fg=col1!40!black}
\setbeamercolor{frametitle}{bg=col2!40!black, fg=white}

\makeatletter
\setlength{\metropolis@progressinheadfoot@linewidth}{2pt}
\setlength{\metropolis@titleseparator@linewidth}{2pt}
\setlength{\metropolis@progressonsectionpage@linewidth}{2pt}
\makeatother
% Boxes
\usepackage[most]{tcolorbox} % Required for boxes
	\tcbuselibrary{skins,breakable,xparse}
\usepackage{varwidth}

\newcounter{definition}
\resetcounteronoverlays{definition}

\renewtcolorbox[auto counter]{definition}[1][]{standard jigsaw,enhanced,sharp corners,frame hidden,boxrule=0pt,breakable,colback=col1!20!white,fonttitle=\bfseries,coltitle=col1!50!black,colframe=col1!50!white,title=Definition~\thedefinition\quad#1\newline,attach title to upper,borderline west={2pt}{0pt}{col1!80!black},left=3mm,phantom={\global\refstepcounter{definition}}}

\newcounter{example}
\resetcounteronoverlays{example}

\renewtcolorbox[auto counter]{example}[1][]{standard jigsaw,enhanced,sharp corners,frame hidden,boxrule=0pt,breakable,colback=col2!20!white,fonttitle=\bfseries,coltitle=col2!50!black,colframe=col2!50!white,title=Example~\theexample\quad#1\newline,attach title to upper,borderline west={2pt}{0pt}{col2!80!black},left=3mm,phantom={\global\refstepcounter{example}}}

\renewtcolorbox{solution}[1][height=4cm]{standard jigsaw,enhanced,sharp corners,frame hidden,boxrule=0pt,breakable,#1,colback=col2!80!white,fonttitle=\bfseries,coltitle=col2!50!black,colframe=col2!50!white,opacityback=.1,title=Solution\newline,attach title to upper,borderline west={2pt}{0pt}{col2!80!black},left=3mm,top=3mm,,borderline north={1pt}{0pt}{col2!80!black},before={\vspace{-7pt}}}

\renewtcolorbox{note}{standard jigsaw,enhanced,sharp corners,frame hidden,boxrule=0pt,breakable,colback=col3!20!white,fonttitle=\normalfont\bfseries,coltitle=col3!50!black,colframe=col3!50!white,title=Note~,attach title to upper,borderline west={2pt}{0pt}{col3!80!black},left=3mm,fontupper=\itshape}

\newtcolorbox{important}{standard jigsaw,enhanced,sharp corners,frame hidden,boxrule=0pt,breakable,colback=col4!20!white,fonttitle=\bfseries,coltitle=col4!50!black,colframe=col4!50!white,title=Important Note\newline,attach title to upper,borderline west={2pt}{0pt}{col4!80!black},left=3mm}

\newtcolorbox{further}{standard jigsaw,enhanced,sharp corners,frame hidden,boxrule=0pt,breakable,colback=col5!20!white,fonttitle=\bfseries,coltitle=col5!50!black,colframe=col5!50!white,title=Further Exercises\newline,attach title to upper,borderline west={2pt}{0pt}{col5!80!black},left=3mm}

\newtcolorbox{outcome}{standard jigsaw,enhanced,sharp corners,frame hidden,boxrule=0pt,breakable,colback=col6!20!white,fonttitle=\bfseries,coltitle=col6!50!black,colframe=col6!50!white,title=Learning Outcome\newline,attach title to upper,borderline west={2pt}{0pt}{col6!80!black},left=3mm}

\newtcolorbox{law}[1][]{standard jigsaw,enhanced,sharp corners,frame hidden,boxrule=0pt,breakable,colback=col3!20!white,fonttitle=\bfseries,coltitle=col1!50!black,colframe=col1!50!white,title=#1~Law\newline,attach title to upper,borderline west={2pt}{0pt}{col1!80!black},borderline north={2pt}{0pt}{col1!80!black},left=3mm,top=3mm}

\renewtcolorbox{theorem}[1][]{standard jigsaw,enhanced,sharp corners,frame hidden,boxrule=0pt,breakable,colback=col3!20!white,fonttitle=\bfseries,coltitle=col2!50!black,colframe=col2!50!white,title=#1~Theorem\newline,attach title to upper,borderline west={2pt}{0pt}{col2!80!black},borderline north={2pt}{0pt}{col2!80!black},left=3mm,top=3mm}

\newtcolorbox{result}{standard jigsaw,enhanced,sharp corners,frame hidden,boxrule=0pt,breakable,colback=col3!20!white,fonttitle=\bfseries,coltitle=col4!50!black,colframe=col4!50!white,title=Result\newline,attach title to upper,borderline west={2pt}{0pt}{col4!80!black},borderline north={2pt}{0pt}{col4!80!black},left=3mm,top=3mm}

\renewtcolorbox{proof}[1][height=4cm]{standard jigsaw,enhanced,sharp corners,frame hidden,boxrule=0pt,breakable,#1,colback=col3!80!white,fonttitle=\bfseries,coltitle=col4!50!black,colframe=col3!50!white,opacityback=.1,title=Proof\newline,attach title to upper,borderline west={2pt}{0pt}{col4!80!black},left=3mm,top=3mm,borderline north={1pt}{0pt}{col4!80!black},before={\vspace{-7pt}}}

\usepackage{environ}

\title{Formulae}
\subtitle{Standard}
\author{MS-A1 Formulae and Equations}
\usepackage[style=iso]{datetime2}
\date{updated: \today
}

\begin{document}

\begin{frame}{Revision}
\begin{tasks}(1)
    \task \textbf{General 2 2014 Q26c}\\
    Solve the equation $\dfrac{5x+1}{3}-4=5-7x$.\hfill\textbf{3}
    \task \textbf{General 2 2018 Q28b}\\
    Solve the equation $\quad\dfrac{2x}{5}+1=\dfrac{3x+1}{2},\quad$ leaving your answer as a fraction.\hfill\textbf{3}
\end{tasks}\vspace{7pt}\pause
  \begin{solution}[]
    \begin{tasks}(2)
    \task \pause$x=1$
    \task \pause$x=\dfrac{5}{11}$
    \end{tasks}
    \end{solution}
\end{frame}

\frame{\titlepage}

\begin{frame}
  \begin{outcome}
    \textbf{Topic:}
    \begin{itemize}
      \item[] Formulae
    \end{itemize}

    \textbf{Syllabus:}
    \begin{itemize}
      \item review evaluating the subject of a formula, given the value of other pronumerals in the formula
    \end{itemize}

    \textbf{Activities/Tasks:}
    \begin{itemize}
      \item Cambridge Ex 3D Q1-27
    \end{itemize}
  \end{outcome}
\end{frame}

\begin{frame}{Equations}
  \begin{definition}[]
    A \textbf{formula} is an equation which connects two or more variables. The plural of formula is \textbf{formulae} or \textbf{formulas}.
  \end{definition}\pause
  Generally a formula is written with a single variable, the \textbf{subject} on the LHS and the rest on the right.
  $$\underset{\overset{\uparrow}{\text{subject}}}{A}=\frac{1}{2}ab\cos{C}$$
\end{frame}

\begin{frame}{Formula substitution}
    If the formula contains two or more variables and we know the value of all but one of them, we can solve an equation to find the remaining variable.\pause
        {\setlength\leftmargini{7ex}
    \begin{enumerate}
    	\item[\textit{Step 1}:] State the variables in the formula and the known values.\pause
    	\item[\textit{Step 2}:] Write down the formula and Substitute the known values into the formula.\pause
    	\item[\textit{Step 3}:] Solve the one variable equation to find the unknown value.
    \end{enumerate}}
\end{frame}

\begin{frame}
  \begin{example}
    The area of a trapezium is given by $\ \ A=\dfrac{h}{2}(a+b)\ \ $ where $a$ and $b$ are the lengths of the parallel sides and $h$ is the height.
    
    Calculate the height of a 20 cm$^2$ trapezium with parallel lengths of 3 cm and 7 cm.
  \end{example}\pause
  \begin{solution}[]
    \vspace{-3em}
    \begin{columns}[t]
      \begin{column}{.3\textwidth}
      \[
      \begin{aligned}
        A&=\onslide<3->{20\text{ cm}^2}\\
        h&=\onslide<4->{?}\\
        a&=\onslide<5->{3\text{ cm}}\\
        b&=\onslide<6->{7\text{ cm}}
      \end{aligned}\pause\pause\pause\pause\pause
      \]
    \end{column}
    \begin{column}{.5\textwidth}
      \[
      \begin{aligned}
        A&=\frac{h}{2}(a+b)\\\pause
        20&=\frac{h}{2}(3+7)\\\pause
        20&=\frac{h}{2}(10)\\\pause
        20&=5h\\\pause
        20\textcolor{col4}{\,\div\,5}&=5h\textcolor{col4}{\,\div\,5}\\\pause
        4&=h\\\pause
        h&=4\pause\text{ cm}
      \end{aligned}
      \]
    \end{column}
  \end{columns}
  The height of the trapezium is equal to 4 cm.
  \end{solution}
\end{frame}

\begin{frame}{Today's work}
  \begin{itemize} 
    \item Cambridge Ex 3D Q1-27
  \end{itemize}
\end{frame}

\end{document}