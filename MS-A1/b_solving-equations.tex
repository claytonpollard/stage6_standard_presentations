\documentclass[aspectratio=169,10pt]{beamer}

\usefonttheme[onlymath
]{serif}

\usetheme[progressbar=head,numbering=none]{metropolis}
\beamertemplatenavigationsymbolsempty
%%\setbeamertemplate{background}[grid]

\usepackage{tasks}
\usepackage{cancel}
\usepackage{multicol}
\usepackage{mathtools}

\definecolor{col1}{HTML}{3A86FF}
\definecolor{col2}{HTML}{0BBF7D}
\definecolor{col3}{HTML}{FFBE0B}
\definecolor{col4}{HTML}{FF006E}
\definecolor{col5}{HTML}{733907}
\definecolor{col6}{HTML}{B340D7}

\setbeamercolor{progress bar}{fg=col2,bg=col3}
\setbeamercolor{title}{fg=col1!40!black}
\setbeamercolor{frametitle}{bg=col2!40!black, fg=white}

\makeatletter
\setlength{\metropolis@progressinheadfoot@linewidth}{2pt}
\setlength{\metropolis@titleseparator@linewidth}{2pt}
\setlength{\metropolis@progressonsectionpage@linewidth}{2pt}
\makeatother
% Boxes
\usepackage[most]{tcolorbox} % Required for boxes
	\tcbuselibrary{skins,breakable,xparse}
\usepackage{varwidth}

\newcounter{definition}
\resetcounteronoverlays{definition}

\renewtcolorbox[auto counter]{definition}[1][]{standard jigsaw,enhanced,sharp corners,frame hidden,boxrule=0pt,breakable,colback=col1!20!white,fonttitle=\bfseries,coltitle=col1!50!black,colframe=col1!50!white,title=Definition~\thedefinition\quad#1\newline,attach title to upper,borderline west={2pt}{0pt}{col1!80!black},left=3mm,phantom={\global\refstepcounter{definition}}}

\newcounter{example}
\resetcounteronoverlays{example}

\renewtcolorbox[auto counter]{example}[1][]{standard jigsaw,enhanced,sharp corners,frame hidden,boxrule=0pt,breakable,colback=col2!20!white,fonttitle=\bfseries,coltitle=col2!50!black,colframe=col2!50!white,title=Example~\theexample\quad#1\newline,attach title to upper,borderline west={2pt}{0pt}{col2!80!black},left=3mm,phantom={\global\refstepcounter{example}}}

\renewtcolorbox{solution}[1][height=4cm]{standard jigsaw,enhanced,sharp corners,frame hidden,boxrule=0pt,breakable,#1,colback=col2!80!white,fonttitle=\bfseries,coltitle=col2!50!black,colframe=col2!50!white,opacityback=.1,title=Solution\newline,attach title to upper,borderline west={2pt}{0pt}{col2!80!black},left=3mm,top=3mm,,borderline north={1pt}{0pt}{col2!80!black},before={\vspace{-7pt}}}

\renewtcolorbox{note}{standard jigsaw,enhanced,sharp corners,frame hidden,boxrule=0pt,breakable,colback=col3!20!white,fonttitle=\normalfont\bfseries,coltitle=col3!50!black,colframe=col3!50!white,title=Note~,attach title to upper,borderline west={2pt}{0pt}{col3!80!black},left=3mm,fontupper=\itshape}

\newtcolorbox{important}{standard jigsaw,enhanced,sharp corners,frame hidden,boxrule=0pt,breakable,colback=col4!20!white,fonttitle=\bfseries,coltitle=col4!50!black,colframe=col4!50!white,title=Important Note\newline,attach title to upper,borderline west={2pt}{0pt}{col4!80!black},left=3mm}

\newtcolorbox{further}{standard jigsaw,enhanced,sharp corners,frame hidden,boxrule=0pt,breakable,colback=col5!20!white,fonttitle=\bfseries,coltitle=col5!50!black,colframe=col5!50!white,title=Further Exercises\newline,attach title to upper,borderline west={2pt}{0pt}{col5!80!black},left=3mm}

\newtcolorbox{outcome}{standard jigsaw,enhanced,sharp corners,frame hidden,boxrule=0pt,breakable,colback=col6!20!white,fonttitle=\bfseries,coltitle=col6!50!black,colframe=col6!50!white,title=Learning Outcome\newline,attach title to upper,borderline west={2pt}{0pt}{col6!80!black},left=3mm}

\newtcolorbox{law}[1][]{standard jigsaw,enhanced,sharp corners,frame hidden,boxrule=0pt,breakable,colback=col3!20!white,fonttitle=\bfseries,coltitle=col1!50!black,colframe=col1!50!white,title=#1~Law\newline,attach title to upper,borderline west={2pt}{0pt}{col1!80!black},borderline north={2pt}{0pt}{col1!80!black},left=3mm,top=3mm}

\renewtcolorbox{theorem}[1][]{standard jigsaw,enhanced,sharp corners,frame hidden,boxrule=0pt,breakable,colback=col3!20!white,fonttitle=\bfseries,coltitle=col2!50!black,colframe=col2!50!white,title=#1~Theorem\newline,attach title to upper,borderline west={2pt}{0pt}{col2!80!black},borderline north={2pt}{0pt}{col2!80!black},left=3mm,top=3mm}

\newtcolorbox{result}{standard jigsaw,enhanced,sharp corners,frame hidden,boxrule=0pt,breakable,colback=col3!20!white,fonttitle=\bfseries,coltitle=col4!50!black,colframe=col4!50!white,title=Result\newline,attach title to upper,borderline west={2pt}{0pt}{col4!80!black},borderline north={2pt}{0pt}{col4!80!black},left=3mm,top=3mm}

\renewtcolorbox{proof}[1][height=4cm]{standard jigsaw,enhanced,sharp corners,frame hidden,boxrule=0pt,breakable,#1,colback=col3!80!white,fonttitle=\bfseries,coltitle=col4!50!black,colframe=col3!50!white,opacityback=.1,title=Proof\newline,attach title to upper,borderline west={2pt}{0pt}{col4!80!black},left=3mm,top=3mm,borderline north={1pt}{0pt}{col4!80!black},before={\vspace{-7pt}}}

\usepackage{environ}

\title{Solving Equations}
\subtitle{Standard}
\author{MS-A1 Formulae and Equations}
\usepackage[style=iso]{datetime2}
\date{updated: \today
}

\begin{document}

\begin{frame}{Revision}
  If $a=4$, $b=-1$, and $c=-3$, evaluate:
  \begin{tasks}(4)
  \task $3a$
  \task $2a-3b$
  \task $2ac$
  \task $ac-3c$
  \task $\dfrac{a}{2b}$
  \task $\dfrac{2ac}{b-c}$
  \task $(2c)^2$
  \task $2c^2$
  \end{tasks}\vspace{7pt}\pause
  \begin{solution}[]
    \begin{tasks}(4)
    \task \pause$12$
    \task \pause$11$
    \task \pause$-24$
    \task \pause$-3$
    \task \pause$-2$
    \task \pause$-12$
    \task \pause$36$
    \task \pause$18$
    \end{tasks}
    \end{solution}
\end{frame}

\frame{\titlepage}

\begin{frame}
  \begin{outcome}
    \textbf{Topic:}
    \begin{itemize}
      \item[] Solving Equations
    \end{itemize}

    \textbf{Syllabus:}
    \begin{itemize}
      \item develop and solve linear equations, including those derived from substituting values into a formula, or those developed from a word description
    \end{itemize}

    \textbf{Activities/Tasks:}
    \begin{itemize}
      \item Cambridge Ex 3C Q1-22
    \end{itemize}
  \end{outcome}
\end{frame}

\begin{frame}{Equations}
  \begin{definition}[Equation]
    A statement asserting that two \textit{expressions} are equal.
  \end{definition}\pause
  \begin{center}
    $\underset{\onslide<3->{\overset{\uparrow}{\text{expression 1}}}}{3n-3}\overset{\onslide<5->{\underset{\downarrow}{\text{is equal to}}}}{=}\underset{\onslide<4->{\overset{\uparrow}{\text{expression 2}}}}{n+1}\quad$ is an equation.
  \end{center}
\end{frame}

\begin{frame}{Solving equations}
  \begin{definition}[]
    To \textbf{solve} an equation means to find the value of the unknown.
  \end{definition}\pause
  In order to solve an equation we need to rearrange the equation to \textbf{isolate} the unknown.\pause We do this by unpacking the equation using \textbf{inverse operations}.\pause The inverse operations are performed on \textit{both sides} of the equation to \textbf{maintain the balance.}\pause

  Once you have found a solution, you can check it is correct by \textbf{substituting} back into the original equation.
\end{frame}

\begin{frame}
  \begin{example}
    Solve for $x$:
    \begin{tasks}(4)
      \task $2x-3=5$
      \task $8-4x=-2$
      \task $\dfrac{x}{4}+7=5$
      \task $\dfrac{1}{3}(x+2)=6$
    \end{tasks}
  \end{example}\pause
  \begin{solution}[]
    \vspace{-1em}
    \begin{columns}[t]
      \begin{column}{.4\textwidth}
    a) $ $\vspace{-1em}\[
      \begin{aligned}
         2x-3&=5\\\pause
         2x-3\textcolor{col4}{\,+\,3}&=5\textcolor{col4}{\,+\,3}\\\pause
         2x&=8\\\pause
         2x\textcolor{col4}{\,\div\,2}&=8\textcolor{col4}{\,\div\,2}\\\pause
         x&=4\pause
      \end{aligned}
      \]
    \end{column}
    \begin{column}{.4\textwidth}
    b) $ $\vspace{-2em}\[
      \begin{aligned}
         8-4x&=-2\\\pause
         8-4x\textcolor{col4}{\,+\,4x\onslide<9->{+2}}&=-2\textcolor{col4}{\,+\,4x\onslide<9->{+2}}\\\pause\pause
         10&=4x\\\pause
         10\textcolor{col4}{\,\div\,4}&=4x\textcolor{col4}{\,\div\,4}\\\pause
         \frac{10}{4}&=x\\\pause\pause
         x&=\frac{10}{4}=\frac{5}{2}=2.5
      \end{aligned}
      \]
    \end{column}
  \end{columns}
  \end{solution}
  \vspace{-7pt}
  \onslide<12->{
  \begin{note}
  It is best practice to have the unknown on the left hand side (LHS) in your answer.
  \end{note}
  }
\end{frame}

\addtocounter{example}{-1}
\begin{frame}
  \begin{example}
    Solve for $x$:
    \begin{tasks}(4)
      \task $2x-3=5$
      \task $8-4x=-2$
      \task $\dfrac{x}{4}+7=5$
      \task $\dfrac{1}{3}(x+2)=6$
    \end{tasks}
  \end{example}
  \begin{solution}[]
    \vspace{-1em}
    \begin{columns}[t]
    \begin{column}{.4\textwidth}
    b) $ $\vspace{-2em}\[
      \begin{aligned}
         8-4x&=-2\\
         8-4x\textcolor{col4}{\,+\,4x\,+\,2}&=-2\textcolor{col4}{\,+\,4x\,+\,2}\\
         10&=4x\\
         10\textcolor{col4}{\,\div\,4}&=4x\textcolor{col4}{\,\div\,4}\\
         \frac{10}{4}&=x\\
         x&=\frac{10}{4}=\frac{5}{2}=2.5
      \end{aligned}
      \]
    \end{column}
      \begin{column}{.4\textwidth}
    c) $ $\vspace{-1em}\[
      \begin{aligned}
         \frac{x}{4}+7&=5\\\pause
         \frac{x}{4}+7\textcolor{col4}{\,-\,7}&=5\textcolor{col4}{\,-\,7}\\\pause
         \frac{x}{4}&=-2\\\pause
         \frac{x}{4}\textcolor{col4}{\,\times\,4}&=-2\textcolor{col4}{\,\times\,4}\\\pause
         x&=-8
      \end{aligned}
      \]
    \end{column}
  \end{columns}
  \end{solution}
  \vspace{-7pt}
    \begin{note}
  It is best practice to have the unknown on the left hand side (LHS) in your answer.
  \end{note}
\end{frame}

\addtocounter{example}{-1}
\begin{frame}
  \begin{example}
    Solve for $x$:
    \begin{tasks}(4)
      \task $2x-3=5$
      \task $8-4x=-2$
      \task $\dfrac{x}{4}+7=5$
      \task $\dfrac{1}{3}(x+2)=6$
    \end{tasks}
  \end{example}
  \begin{solution}[]
    \vspace{-1em}
    \begin{columns}[t]
      \begin{column}{.4\textwidth}
    c) $ $\vspace{-1em}\[
      \begin{aligned}
         \frac{x}{4}+7&=5\\
         \frac{x}{4}+7\textcolor{col4}{\,-\,7}&=5\textcolor{col4}{\,-\,7}\\
         \frac{x}{4}&=-2\\
         \frac{x}{4}\textcolor{col4}{\,\times\,4}&=-2\textcolor{col4}{\,\times\,4}\\
         x&=-8
      \end{aligned}
      \]
    \end{column}
      \begin{column}{.4\textwidth}
    d) $ $\vspace{-2em}\[
      \begin{aligned}
         \frac{1}{3}(x+2)&=6\\\pause
         \frac{1}{3}(x+2)\textcolor{col4}{\,\times\,3}&=6\textcolor{col4}{\,\times\,3}\\\pause
         x+2&=18\\\pause
         x+2\textcolor{col4}{\,-\,2}&=18\textcolor{col4}{\,-\,2}\\\pause
         x&=16
      \end{aligned}
      \]
    \end{column}
  \end{columns}
  \end{solution}
\end{frame}

\begin{frame}{Repeated unknowns}\pause
If the unknown appears in the equation more than once, we follow these steps:\pause
\begin{enumerate}
	\item Expand brackets and collect like terms.\pause
	\item If the unknown appears on both sides of the equation, remove it from one side using inverse operations.\pause
	\item Isolate the unknown and solve the equation.
\end{enumerate}
\end{frame}

\begin{frame}
  \begin{example}
    Solve $\quad4(2x+5)-3(x-2)=16\quad$ for $x$.
  \end{example}\pause
  \begin{solution}[]
    \[
      \begin{aligned}
         4(2x+5)-3(x-2)&=16\\\pause
         8x+20-3x+6&=16\\\pause
         5x+26&=16\\\pause
         5x+26\textcolor{col4}{\,-\,26}&=16\textcolor{col4}{\,-\,26}\\\pause
         5x&=-10\\\pause
         5x\textcolor{col4}{\,\div\,5}&=-10\textcolor{col4}{\,\div\,5}\\\pause
         x&=-2
      \end{aligned}
      \]
  \end{solution}
\end{frame}

\begin{frame}
  \begin{example}
    Solve for $x$:
    \begin{tasks}(2)
      \task $4x-3=3x+7$
      \task $5-3(-1+x)=x$
    \end{tasks}
  \end{example}\pause
  \begin{solution}[]
    \vspace{-1em}
    \begin{columns}[t]
      \begin{column}{.4\textwidth}
    a) $ $\vspace{-2em}\[
      \begin{aligned}
        4x-3&=3x+7\\\pause
        4x\textcolor{col4}{\,-\,3x}-3\textcolor{col4}{\,+\,3}&=3x\textcolor{col4}{\,-\,3x}+7\textcolor{col4}{\,+\,3}\\\pause
        x&=10\pause
      \end{aligned}
      \]
    \end{column}
      \begin{column}{.4\textwidth}
    b) $ $\vspace{-2em}\[
      \begin{aligned}
        5-3(-1+x)&=x\\\pause
        5+3-3x&=x\\\pause
        8-3x\textcolor{col4}{\,+\,3x}&=x\textcolor{col4}{\,+\,3x}\\\pause
        8&=4x\\\pause
        8\textcolor{col4}{\,\div\,4}&=4x\textcolor{col4}{\,\div\,4}\\\pause
        2&=x\\\pause
        x&=2
      \end{aligned}
      \]
    \end{column}
  \end{columns}
  \end{solution}
\end{frame}

\begin{frame}{Fractions}
  For equations involving fractions, first write all fractions with a \textbf{common denominator}.\pause
  
  For example:
    \begin{itemize}
      \item In $\dfrac{2x}{3}=\dfrac{x}{4}$ the lowest common denominator is \pause $\ \ 12$.
      \item In $\dfrac{3}{x}=\dfrac{5}{2x-1}$ the lowest common denominator is \pause $\ \ x(2x-1)$.
    \end{itemize}
  Once the fractions have common denominators you can \textbf{equate the numerators}
\end{frame}

\begin{frame}
\small
  \begin{example}
    Solve $\quad\dfrac{2-x}{3}=\dfrac{x}{5}\quad$ for $x$.
  \end{example}
  \begin{solution}[]\vspace{-2em}
    \[
      \begin{aligned}
         \frac{2-x}{3}&=\frac{x}{5}\\\pause
         \frac{2-x}{3}\textcolor{col4}{\,\times\,\frac{5}{5}}&=\frac{x}{5}\textcolor{col4}{\,\times\,\frac{3}{3}}\\\pause
         \frac{5(2-x)}{15}&=\frac{3x}{15}\\\pause
         5(2-x)&=3x&\text{(equating the numerators)}\\\pause
         10-5x&=3x\\\pause
         10-5x\textcolor{col4}{\,+\,5x}&=3x\textcolor{col4}{\,+\,5x}\\\pause
         10&=8x\\\pause
         10\textcolor{col4}{\,\div\,8}&=8x\textcolor{col4}{\,\div\,8}\\\pause
         \frac{10}{8}&=x\\\pause
         x&=\frac{10}{8}=\frac{5}{4}=1.25
      \end{aligned}
      \]
  \end{solution}
\end{frame}

\begin{frame}
\small
  \begin{example}
    Solve $\quad\dfrac{7}{x+1}=\dfrac{3}{x}\quad$ for $x$.
  \end{example}
  \begin{solution}[]\vspace{-2em}
    \[
      \begin{aligned}
         \frac{7}{x+1}&=\frac{3}{x}\\\pause
         \frac{7}{x+1}\textcolor{col4}{\,\times\,\frac{x}{x}}&=\frac{3}{x}\textcolor{col4}{\,\times\,\frac{x+1}{x+1}}\\\pause
         \frac{7x}{x(x+1)}&=\frac{3(x+1)}{x(x+1)}\\\pause
         7x&=3(x+1)&\text{(equating the numerators)}\\\pause
         7x&=3x+3\\\pause
         7x\textcolor{col4}{\,-\,3x}&=3x\textcolor{col4}{\,-\,3x}+3\\\pause
         4x&=3\\\pause
         4x\textcolor{col4}{\,\div\,4}&=3\textcolor{col4}{\,\div\,4}\\\pause
         x&=\frac{3}{4}=0.75\\
      \end{aligned}
      \]
  \end{solution}
\end{frame}

\begin{frame}{Today's work}
  \begin{itemize} 
    \item Cambridge Ex 3C Q1-22
  \end{itemize}
\end{frame}

\end{document}