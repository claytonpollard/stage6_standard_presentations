\documentclass[aspectratio=169,10pt]{beamer}

\usefonttheme[onlymath
]{serif}

\usetheme[progressbar=head,numbering=none]{metropolis}
\beamertemplatenavigationsymbolsempty
%%\setbeamertemplate{background}[grid]

\usepackage{tasks}
\usepackage{cancel}
\usepackage{multicol}
\usepackage{mathtools}

\definecolor{col1}{HTML}{3A86FF}
\definecolor{col2}{HTML}{0BBF7D}
\definecolor{col3}{HTML}{FFBE0B}
\definecolor{col4}{HTML}{FF006E}
\definecolor{col5}{HTML}{733907}
\definecolor{col6}{HTML}{B340D7}

\setbeamercolor{progress bar}{fg=col2,bg=col3}
\setbeamercolor{title}{fg=col1!40!black}
\setbeamercolor{frametitle}{bg=col2!40!black, fg=white}

\makeatletter
\setlength{\metropolis@progressinheadfoot@linewidth}{2pt}
\setlength{\metropolis@titleseparator@linewidth}{2pt}
\setlength{\metropolis@progressonsectionpage@linewidth}{2pt}
\makeatother
% Boxes
\usepackage[most]{tcolorbox} % Required for boxes
	\tcbuselibrary{skins,breakable,xparse}
\usepackage{varwidth}

\newcounter{definition}
\resetcounteronoverlays{definition}

\renewtcolorbox[auto counter]{definition}[1][]{standard jigsaw,enhanced,sharp corners,frame hidden,boxrule=0pt,breakable,colback=col1!20!white,fonttitle=\bfseries,coltitle=col1!50!black,colframe=col1!50!white,title=Definition~\thedefinition\quad#1\newline,attach title to upper,borderline west={2pt}{0pt}{col1!80!black},left=3mm,phantom={\global\refstepcounter{definition}}}

\newcounter{example}
\resetcounteronoverlays{example}

\renewtcolorbox[auto counter]{example}[1][]{standard jigsaw,enhanced,sharp corners,frame hidden,boxrule=0pt,breakable,colback=col2!20!white,fonttitle=\bfseries,coltitle=col2!50!black,colframe=col2!50!white,title=Example~\theexample\quad#1\newline,attach title to upper,borderline west={2pt}{0pt}{col2!80!black},left=3mm,phantom={\global\refstepcounter{example}}}

\renewtcolorbox{solution}[1][height=4cm]{standard jigsaw,enhanced,sharp corners,frame hidden,boxrule=0pt,breakable,#1,colback=col2!80!white,fonttitle=\bfseries,coltitle=col2!50!black,colframe=col2!50!white,opacityback=.1,title=Solution\newline,attach title to upper,borderline west={2pt}{0pt}{col2!80!black},left=3mm,top=3mm,,borderline north={1pt}{0pt}{col2!80!black},before={\vspace{-7pt}}}

\renewtcolorbox{note}{standard jigsaw,enhanced,sharp corners,frame hidden,boxrule=0pt,breakable,colback=col3!20!white,fonttitle=\normalfont\bfseries,coltitle=col3!50!black,colframe=col3!50!white,title=Note~,attach title to upper,borderline west={2pt}{0pt}{col3!80!black},left=3mm,fontupper=\itshape}

\newtcolorbox{important}{standard jigsaw,enhanced,sharp corners,frame hidden,boxrule=0pt,breakable,colback=col4!20!white,fonttitle=\bfseries,coltitle=col4!50!black,colframe=col4!50!white,title=Important Note\newline,attach title to upper,borderline west={2pt}{0pt}{col4!80!black},left=3mm}

\newtcolorbox{further}{standard jigsaw,enhanced,sharp corners,frame hidden,boxrule=0pt,breakable,colback=col5!20!white,fonttitle=\bfseries,coltitle=col5!50!black,colframe=col5!50!white,title=Further Exercises\newline,attach title to upper,borderline west={2pt}{0pt}{col5!80!black},left=3mm}

\newtcolorbox{outcome}{standard jigsaw,enhanced,sharp corners,frame hidden,boxrule=0pt,breakable,colback=col6!20!white,fonttitle=\bfseries,coltitle=col6!50!black,colframe=col6!50!white,title=Learning Outcome\newline,attach title to upper,borderline west={2pt}{0pt}{col6!80!black},left=3mm}

\newtcolorbox{law}[1][]{standard jigsaw,enhanced,sharp corners,frame hidden,boxrule=0pt,breakable,colback=col3!20!white,fonttitle=\bfseries,coltitle=col1!50!black,colframe=col1!50!white,title=#1~Law\newline,attach title to upper,borderline west={2pt}{0pt}{col1!80!black},borderline north={2pt}{0pt}{col1!80!black},left=3mm,top=3mm}

\renewtcolorbox{theorem}[1][]{standard jigsaw,enhanced,sharp corners,frame hidden,boxrule=0pt,breakable,colback=col3!20!white,fonttitle=\bfseries,coltitle=col2!50!black,colframe=col2!50!white,title=#1~Theorem\newline,attach title to upper,borderline west={2pt}{0pt}{col2!80!black},borderline north={2pt}{0pt}{col2!80!black},left=3mm,top=3mm}

\newtcolorbox{result}{standard jigsaw,enhanced,sharp corners,frame hidden,boxrule=0pt,breakable,colback=col3!20!white,fonttitle=\bfseries,coltitle=col4!50!black,colframe=col4!50!white,title=Result\newline,attach title to upper,borderline west={2pt}{0pt}{col4!80!black},borderline north={2pt}{0pt}{col4!80!black},left=3mm,top=3mm}

\renewtcolorbox{proof}[1][height=4cm]{standard jigsaw,enhanced,sharp corners,frame hidden,boxrule=0pt,breakable,#1,colback=col3!80!white,fonttitle=\bfseries,coltitle=col4!50!black,colframe=col3!50!white,opacityback=.1,title=Proof\newline,attach title to upper,borderline west={2pt}{0pt}{col4!80!black},left=3mm,top=3mm,borderline north={1pt}{0pt}{col4!80!black},before={\vspace{-7pt}}}

\usepackage{environ}

\title{Rearranging Formulae}
\subtitle{Standard}
\author{MS-A1 Formulae and Equations}
\usepackage[style=iso]{datetime2}
\date{updated: \today
}

\begin{document}

\begin{frame}{Revision}
The volume for a cylinder is given by the formula $\ \ V=\pi r^2h\ \ $ where $r$ is the radius, and $h$ is the height. Find the following to three significant figures.
\begin{tasks}(1)
\task the volume of a cylindrical tin can of radius $10$ cm and height $15$ cm
\task the height of a cylinder of radius $4$ cm if its volume is $60$ cm$^3$
\task the radius, in cm, of cable with volume $50$ cm$^3$ and length $45$ cm.
\end{tasks}\vspace{7pt}\pause
  \begin{solution}[]
    \begin{tasks}(3)
    \task \pause$4710$ cm$^3$
    \task \pause$1.19$ cm
    \task \pause$0.595$ cm
    \end{tasks}
    \end{solution}
\end{frame}

\frame{\titlepage}

\begin{frame}
  \begin{outcome}
    \textbf{Topic:}
    \begin{itemize}
      \item[] Rearranging Formulae
    \end{itemize}

    \textbf{Syllabus:}
    \begin{itemize}
      \item change the subject of a formula
    \end{itemize}

    \textbf{Activities/Tasks:}
    \begin{itemize}
      \item Cambridge Ex 3E Q1-21
    \end{itemize}
  \end{outcome}
\end{frame}

\begin{frame}{Formula Rearrangement}
  By performing operations to both sides of a formula, they can be \textbf{rearranged} to make \textbf{equivalent} formulae where other variables are the subjects.\pause

  For example
  \begin{columns}[t]
    \begin{column}{.4\textwidth}
      \[
      \begin{aligned}
        V&=\pi r^2h\\\pause
        V\textcolor{col4}{\,\div\,\pi r^2}&=\pi r^2h\textcolor{col4}{\,\div\,\pi r^2}\\\pause
        \frac{V}{\pi r^2}&=h\\\pause
        h&=\frac{V}{\pi r^2}\pause
      \end{aligned}
      \]
    \end{column}
    \begin{column}{.4\textwidth}
      \[
      \begin{aligned}
        V&=\pi r^2h\\\pause
        V\textcolor{col4}{\,\div\,\pi h}&=\pi r^2h\textcolor{col4}{\,\div\,\pi h}\\\pause
        \frac{V}{\pi h}&=r^2\\\pause
        r^2&=\frac{V}{\pi h}\\\pause
        \textcolor{col4}{\sqrt{\textcolor{black}{r^2}}}&=\textcolor{col4}{\sqrt{\textcolor{black}{\frac{V}{\pi h}}}}\\\pause
        r&=\sqrt{\frac{V}{\pi h}}&(r>0)
      \end{aligned}
      \]
    \end{column}
  \end{columns}
\end{frame}

\begin{frame}
  \begin{example}
    Make $a$ the subject of $\quad2a-7b=23$.
  \end{example}\pause
  \begin{solution}[]
    \[
    \begin{aligned}
      2a-7b&=23\\\pause
      2a-7b\textcolor{col4}{\,+\,7b}&=23\textcolor{col4}{\,+\,7b}\\\pause
      2a&=23-7b\\\pause
      \textcolor{col4}{\frac{\textcolor{black}{2a}}{2}}&=\textcolor{col4}{\frac{\textcolor{black}{23-7b}}{2}}\\\pause
      a&=\frac{23-7b}{2}
    \end{aligned}
    \]
  \end{solution}
\end{frame}

\begin{frame}{Rearranging then substituting}
Previously, during formula substitution, the variables were replaced by numbers and then the equation was solved. However, often we need to substitute several values for the unknowns and solve the equation for each case. In this situation it is quicker to \textbf{rearrange} the formula \textbf{before substituting}.
\end{frame}

\begin{frame}
\small
  \begin{example}
    The surface area of a sphere is given by $\ \ A=4\pi r^2\ \ $ where $r$ is the sphere's radius.
    \begin{tasks}(1)
      \task Rearrange this formula to make $r$ the subject.
      \task Hence find the radius to 3 significant figures when the surface area is:
    \end{tasks}
    \begin{tasks}(3)
        \task[i)] 10 cm$^2$
        \task[ii)] 20 cm$^2$
        \task[iii)] 30 cm$^2$
    \end{tasks}
  \end{example}\pause
  \begin{solution}[]
  \vspace{-1em}
    \begin{columns}[t]
      \begin{column}{.22\textwidth}
      a)\vspace{-2em}
        \[
        \begin{aligned}
          A&=4\pi r^2\\\pause
          A\textcolor{col4}{\,\div\,4\pi}&=4\pi r^2\textcolor{col4}{\,\div\,4\pi}\\\pause
          \frac{A}{4\pi}&=r^2\\\pause
          \textcolor{col4}{\sqrt{\textcolor{black}{\frac{A}{4\pi}}}}&=\textcolor{col4}{\sqrt{\textcolor{black}{r^2}}}\\\pause
          r&=\sqrt{\frac{A}{4\pi}}&(r>0)\pause
        \end{aligned}
        \]
      \end{column}
      \begin{column}{.2\textwidth}
        \vspace{-2em}
        \[
        \begin{aligned}
          \text{bi) }\  r&=\sqrt{\frac{A}{4\pi}}\\\pause
          r&=\sqrt{\frac{10}{4\pi}}\\\pause
          r&=0.892\text{ cm}\pause
        \end{aligned}
        \]
      \end{column}
      \begin{column}{.2\textwidth}
        \vspace{-2em}
        \[
        \begin{aligned}
          \text{bii) }\  r&=\sqrt{\frac{A}{4\pi}}\\\pause
          r&=\sqrt{\frac{20}{4\pi}}\\\pause
          r&=1.26\text{ cm}\pause
        \end{aligned}
        \]
      \end{column}
      \begin{column}{.2\textwidth}
        \vspace{-2em}
        \[
        \begin{aligned}
          \text{biii) }\  r&=\sqrt{\frac{A}{4\pi}}\\\pause
          r&=\sqrt{\frac{30}{4\pi}}\\\pause
          r&=1.55\text{ cm}
        \end{aligned}
        \]
      \end{column}
    \end{columns}
  \end{solution}
\end{frame}

\begin{frame}{Today's work}
  \begin{itemize} 
    \item Cambridge Ex 3E Q1-21
  \end{itemize}
\end{frame}

\end{document}